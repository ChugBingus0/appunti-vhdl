\documentclass{article}
\usepackage[utf8]{inputenc}
\usepackage[italian]{babel}
\usepackage{verbatim}
\usepackage{geometry}
\usepackage{enumitem}
\usepackage{ccicons}
\usepackage{setspace}
\usepackage{hyperref}
\usepackage{xcolor}

\geometry{left = 2 cm, right = 2 cm, bottom = 2 cm, top = 2.5 cm}
\setlength{\parindent}{0em}
\hypersetup{
    colorlinks,
    linkcolor={black},
    citecolor={black},
    urlcolor={black}
}

\pagestyle{plain}
\pagenumbering{gobble}

\title{\Huge VHDL cheatsheet}
\author{\LARGE Lorenzo Rossi}
\date{\LARGE Anno Accademico 2019/2020}

\begin{document}

\maketitle

\vspace{15em}

\large
\begin{doublespacing}
\centerline{Email: lorenzo14.rossi@mail.polimi.it}
\centerline{GitHub: https://github.com/lorossi/appunti-vhdl}

\vspace{16em}
\centerline{Quest'opera è distribuita con Licenza Creative Commons Attribuzione}
\centerline{Non commerciale 4.0 Internazionale \ccbynceu}
\end{doublespacing}
\newpage


\pagenumbering{roman}

\tableofcontents
\clearpage
\pagenumbering{arabic}
\newpage

\section{Sintassi base}
\subsection{Case sensitivity}
Il VHDL è \textit{case insensitive}, ovverosia non fa differenza tra lettere maiuscole e minuscole \textbf{in nessun caso.}
\subsection{Assegnazioni}
\subsubsection{Assegnazioni di Signal}
\begin{verbatim}
	    SIGNAL_NAME <= EXPRESSION;
\end{verbatim}
\subsubsection{Assegnazioni di Variable}
\begin{verbatim}
	    VARIABLE NAME := EXPRESSION;
\end{verbatim}
\subsubsection{Slicing di vettori}
\textit{SIGNAL\_0} e \textit{SIGNAL\_1} sono segnali di tipo \textit{std\_logic\_vector}
\begin{verbatim}
	    SIGNAL_1 <= SIGNAL_0(RANGE);
	    SIGNAL_1 <= SIGNAL_0(N downto M);
	    SIGNAL_1 <= SIGNAL_0(M to N);
\end{verbatim}
\subsection{Operatori}
\subsubsection{Uso degli operatori}
\begin{verbatim}
	    SIGNAL_1 <= SIGNAL_2 operatore SIGNAL_3;
\end{verbatim}

\subsubsection{Operatori logici}
\begin{itemize}
	\item not \textit{operazione di negazione}
	\item and \textit{operazione di moltiplicazione logica}
	\item or \textit{operazione di addizione logica}
	\item nor \textit{operazione di somma negata}
	\item nand \textit{operazione di moltiplicazione negata}
	\item xor \textit{or esclusivo}
	\item xnor \textit{or esclusivo negato}
\end{itemize}

\newpage

\subsubsection{Shift}

\textbf{Le operazioni di \textit{shift} funzionano solo sugli \textit{std\_logic\_vector} nella versione di VHDL successiva alla 2008.} Per gli shift tra \textit{signed} o in versioni precedenti, consultare il paragrafo \ref{signedshift}

\begin{itemize}
	\item SLL \textit{Shift left, i bit più a destra sono rimpiazzati da zeri}
	\item SRL \textit{Shift right, i bit più a sinistra sono rimpiazzati da zeri}
	\item SLA \textit{Shift left aritmetico}
	\item SRA \textit{Shift right aritmetico}
	\item ROL \textit{Rotate left}
	\item ROR \textit{Rotate right}
\end{itemize}

\subsubsection{Shift tra signed} \label{signedshift}
Per preservare il \textit{bit di segno} non sarà sufficiente usare le funzioni \textit{SLL} o \textit{SRL} ma bisognerà usare \textit{attributes} (paragrafo \ref{attributes}) e l'operazione di \textit{concatenazione} (paragrafo \ref{miscellanei})

\begin{itemize}
	\item \textit{SHIFT LEFT}
		\begin{verbatim}
		    signal_shifted <= signal(signal'HIGH) & signal(signal'HIGH - 2 downto 0) & '0'
		\end{verbatim}
	\item \textit{SHIFT RIGHT}
		\begin{verbatim}
		    signal_shifted <= signal(signal'HIGH) & signal(signal'HIGH downto 1)
		\end{verbatim}
\end{itemize}

Alternativamente sarà possibile usare le funzioni \textit{built-in} definite nella \textit{numeric\_std} che accettano sia \textit{signed} che \textit{unsigned}:
\begin{itemize}
	\item shift\_left()
	\item shift\_right()
\end{itemize}

Uso delle funzioni:
\begin{verbatim}
    DESTINATION_SIGNAL <= shift_left(SOURCE_SIGNAL, N OF BITS)
    DESTINATION_SIGNAL <= shift_right(SOURCE_SIGNAL, N OF BITS)
\end{verbatim}
\subsubsection{Operatori relazionali}
\begin{itemize}
	\item \(=\) \textit{uguale}
	\item \(/=\) \textit{non uguale, diverso}
	\item \(<\) \textit{minore}
	\item \(>\) \textit{maggiore}
	\item \(<=\) \textit{minore uguale}
	\item \(>=\) \textit{maggiore uguale}
\end{itemize}
\subsubsection{Operatori aritmetici}
\begin{itemize}
	\item \(**\) \textit{elevazione a potenza}
	\item rem \textit{resto}
	\item mod \textit{modulo}
	\item \(/\) \textit{divisione}
	\item \(*\) \textit{moltiplicazione}
	\item \(+\) \textit{somma}
	\item abs \textit{valore assoluto}
	      	
\end{itemize}
Per incrementare di \(1\) una variabile non si può usare l'operatore \(++\) ma bisogna usare l'espressione
\begin{verbatim}
	    SIGNAL <= SIGNAL + 1;
\end{verbatim}
E analogamente bisognerà comportarsi con sottrazione e altri operatori aritmetici.
\subsubsection{Operatori miscellanei} \label{miscellanei}
\begin{itemize}
	\item \& \textit{operazione di concatenazione}
	\item \(--\) \textit{commento in linea}
	\item Others \textit{si riferisce a tutti gli elementi di un signal che non sono già stati menzionati}
	\item array(i) \textit{accesso all'i-esimo elemento dell'array}
	\item array(i, j) \textit{accesso al j-esimo elemento dell'i-esimo elemento dell'array - nested 2D array}
	\item array := (Others \( => \) (Others \( => \) '0')) \textit{inizializzazione di un nested 1D array}
\end{itemize}
	
\newpage

\section{Modeling Styles}
\begin{itemize}
	\item Structural modeling
	      \begin{itemize}
	      	\item Implementazione come unione di porte/strutture
	      	\item Technology dependent
	      \end{itemize}
	\item Dataflow Modeling
	      \begin{itemize}
	      	\item Implementazione come descrizione combinatoria tramite porte logiche di base
	      	\item Technology independent
	      \end{itemize}
	\item Behavioral modeling
	      \begin{itemize}
	      	\item Implementazione come descrizione del comportamento che ha l'entity
	      	\item Non riflette direttamente l'implementazione
	      \end{itemize}
	\item Mixed Modeling
	      \begin{itemize}
	      	\item Combinazione delle precedenti
	      \end{itemize}
\end{itemize}

\section{Librerie}
\begin{itemize}
	\item Definiscono i tipi base o il comportamento delle funzioni elementari
	\item Tutte le librerie devono essere dichiarate manualmente, tranne la libreria \textit{standard} che contiene informazione base e direttive date all'analizzatore (es. \textit{boolean})
	\item Librerie fondamentali:
	      \begin{itemize}
	      	\item \textit{std\_logic\_1164.all} enhanced signal types
	      	\item \textit{numeric\_std.all} numerical computation
	      	\item \textit{math\_real.all} mathematical operations
	      \end{itemize}
\end{itemize}

\newpage

\section{Tipi di dato}
Il linguaggio VHDL contiene diversi tipi di dato che possono caratterizzare \textit{signal} e \textit{variables}
\subsection{Integer}
\begin{itemize}
	\item Definito nella libreria \textit{std\_logic\_1164.all}
	\item Contiene gli interi da \(-2^{31}-1\) a \(2^{31}-1\)
	\item Non ci si può affidare al \textit{roll-up} o all'\textit{overflow}
	\item Si può specificare la larghezza massima dell'intero. \textbf{Tuttavia in tal caso non ci si potrà affidare al \textit{roll-over.}} 
\end{itemize}
\begin{verbatim}
        my_num INTEGER range 0 to 64
\end{verbatim}
\subsection{Natural e Positive}
\begin{itemize}
	\item \`Sono \textit{subtype} di \textit{INTEGER}
	\item Non sono di 32 bit
	\begin{itemize}
		\item Il \textit{natural} contiene gli interi da \(0\) a \(2^{31}-1\)
\begin{verbatim}
    subtype NATURAL is INTEGER range 0 to INTEGER'HIGH
 \end{verbatim}
		\item Il \textit{positive} contiene gli interi da \(1\) a \(2^{31}-1\)
\begin{verbatim}
    subtype POSITIVE is INTEGER range 1 to INTEGER'HIGH
 \end{verbatim}
	\end{itemize}	
\end{itemize}
\subsection{Std\_Ulogic}
\begin{itemize}
	\item Definito nella libreria \textit{std\_logic\_1164}
	\item Definisce qualsiasi stato di un elemento ad 1 bit
	      \begin{itemize}
	      	\item 'U' \textit{forcing uninitialized}
	      	\item 'X' \textit{forcing unknown}
	      	\item '0' \textit{forcing 0}
	      	\item '1' \textit{forcing 1}
	      	\item 'Z' \textit{high impedance}
	      	\item 'W' \textit{WEAK unknown}
	      	\item 'L' \textit{WEAK low}
	      	\item 'H' \textit{WEAL high}
	      	\item '-' \textit{don't care}
	      \end{itemize}
\end{itemize}	
     
\subsection{Std\_Logic}
\begin{itemize}
	\item Definito nella libreria \textit{standard}
	\item \`E un \textit{subtype} del tipo \textit{std\_ulogic}
	\item Ovvia automaticamente al problema che sorge quando due signal diversi vengono connessi tra di loro (ad esempio, all'uscita di due buffer)
	\item Il segnale viene automaticamente \textit{risolto} (assegnato) tramite una \textit{tabella di risoluzione}
\end{itemize}

\begin{table}[htb]
\centering
\begin{tabular}{cccccccccc}
\multicolumn{1}{l}{} & \multicolumn{9}{c}{Tabella di risoluzione}                                                                         \\
                     & \textbf{U} & \textbf{X} & \textbf{0} & \textbf{1} & \textbf{Z} & \textbf{W} & \textbf{L} & \textbf{H} & \textbf{-} \\
\textbf{U}           & U          & U          & U          & U          & U          & U          & U          & U          & U          \\
\textbf{X}           & U          & X          & X          & X          & X          & X          & X          & X          & X          \\
\textbf{0}           & U          & X          & 0          & X          & 0          & 0          & 0          & 0          & X          \\
\textbf{1}           & U          & X          & X          & 1          & 1          & 1          & 1          & 1          & X          \\
\textbf{Z}           & U          & X          & 0          & 1          & Z          & W          & L          & H          & X          \\
\textbf{W}           & U          & X          & 0          & 1          & W          & W          & W          & W          & X          \\
\textbf{L}           & U          & X          & 0          & 1          & L          & W          & L          & W          & X          \\
\textbf{H}           & U          & X          & 0          & 1          & H          & W          & W          & H          & X          \\
\textbf{-}           & U          & X          & X          & X          & X          & X          & X          & X          & X         
\end{tabular}
\end{table}

\subsection{Std\_Logic\_Vector}
\begin{itemize}
	\item Estensione di \textit{std\_logic} sotto forma di vettore
\end{itemize}
\begin{verbatim}
    STD_LOGIC_VECTOR(15 downto 0)
    STD_LOGIC_VECTOR(0 to 15)
\end{verbatim}

\subsection{Signed/Unsigned}
\begin{itemize} \item Definiti nella libreria \textit{numeric\_std}
	\item Array di \textit{signed} or \textit{unsigned} con tutte le loro operazioni definite
	\item Non c'è limite alla dimensione che possono assumere
	\item Si può fare affidamento al \textit{roll-up} o all'\textit{overflow}
	\item C'è controllo sulla rappresentazione numerica
\end{itemize}
\begin{verbatim}
    type UNSIGNED is ARRAY (NATURAL range <>) OF std_logic
    type SIGNED is ARRAY (NATURAL range <>) OF std_logic
\end{verbatim}

\newpage

\subsection{Conversioni tra tipi di dato}
\begin{itemize}
	\item Categorizzazione dei tipi
	\begin{itemize}
		\item \textit{Integer} \(\rightarrow\) numero
		\item \textit{Signed, Unsigned, Std\_logic\_vector} \(\rightarrow\) vettori di bit
	\end{itemize}
	\item Conversione dei tipi
	\begin{itemize}
	\item Da \textit{Integer} a:
	\begin{itemize}
		\item \textit{Signed} \(\rightarrow\) \textit{to\_signed(integer, signed'LENGTH)}
		\item \textit{Unsigned} \(\rightarrow\) \textit{to\_unsigned(integer, unsigned'LENGTH)}
	\end{itemize}
	\item Da \textit{Signed} a:
	\begin{itemize}
		\item \textit{Integer} \(\rightarrow\) \textit{to\_integer(signed)}
		\item \textit{Std\_logic\_vector} \(\rightarrow\) \textit{std\_logic\_vector(signed)}
	\end{itemize}
	\item Da \textit{Unsigned} a:
	\begin{itemize}
		\item \textit{Integer}\(\rightarrow\) \textit{to\_integer(unsigned)}
		\item \textit{std\_logic\_vector}\(\rightarrow\) \textit{std\_logic\_vector(unsigned)}
	\end{itemize}
	\item Da \textit{Std\_logic\_vector} a:
	\begin{itemize}
		\item \textit{Signed}\(\rightarrow\) \textit{signed(std\_logic\_vector)}
		\item \textit{Unsigned}\(\rightarrow\) \textit{unsigned(std\_logic\_vector)}
	\end{itemize}
\end{itemize}
	\end{itemize}

\newpage

\section{Basic statements}
\subsection{Entity}
\begin{itemize}
	\item Una \textit{entity} è la descrizione dell'interfaccia tra il design e l'ambiente esterno
	\item Può indicare le \textit{declarations} e gli \textit{statements} che sono parte del design
	\item Una \textit{entity} può essere condivisa tra più elementi del \textit{design}, ognuno con la sua differente architettura
	\item Struttura di una \textit{entity}
	      \begin{verbatim}
    entity ENTITY_NAME is
        port (
            PORT_NAME : PORT_MODE PORT_TYPE;
            PORT_NAME : PORT_MODE PORT_TYPE
            ...
        );
    end ENTITY_NAME;
	      \end{verbatim}
	\item \textbf{Non ci vuole il \textit{punto e virgola} nella dichiarazione dell'ultima porta}
	\item Esempio di una \textit{entity}
	      \begin{verbatim}
    entity or2 is
        port (
            a : in std_logic;
            b : in std_logic;
            c : out std_logic
        );
    end or;
	      \end{verbatim}
	\item \textbf{Si dichiarano prima della \textit{architecture}}
\end{itemize}

\subsection{Architecture}
\begin{itemize}
	\item Dentro la \textit{architecture} viene descritto il comportamento di uno o più componenti istanziati
	\item Si possono usare più \textit{architecture} per \textit{entity}, scritte in forme diverse, per poi scegliere l'implementazione che si desidera
	\item Struttura di una \textit{architecture}
	      \begin{verbatim}
    architecture ARCHITECTURE_NAME of ENTITY_NAME is
    Begin
        ...
    End ARCHITECTURE_NAME
	      \end{verbatim}
	      
\newpage	      
	      
	\item Esempio di una \textit{architecture}
	      \begin{verbatim}
    architecture or2_a of or2 is
    Begin
        c <= a or b;
    End or2_a;
	      \end{verbatim}
\end{itemize}

\subsection{Modes}
\begin{itemize}
	\item Il modo di una porta definisce la direzione dei segnali che la attraversano
	\item Tipi di modi:
	      \begin{itemize}
	      	\item IN \textit{ingresso, può solo essere letta}
	      	\item OUT \textit{uscita, può solo essere scritta}
	      	\item INOUT \textit{sia ingresso che uscita}
	      	\item BUFFER \textit{porta di uscita che può essere usata anche per leggere il valore scritto in precedenza}
	      	\item LINKAGE \textit{porta speciale, crea collegamento diretto senza buffer}
	      \end{itemize}
\end{itemize}

\subsection{Signal}
\begin{itemize}
	\item Oggetti primari per la creazione di un sistema hardware
	\item Ad ogni segnale è associato un nome, un tipo ed un valore iniziale \textit{(facoltativamente)}
	      \begin{itemize}
	      	\item Un segnale non inizializzato avrà valore indefinito \textit{U}
	      \end{itemize}
	\item Possono essere utilizzati per memorizzare valori o per connettere entità
	\item Struttura di un \textit{signal}
	      \begin{verbatim}
    signal SIGNAL_NAME : SIGNAL_TYPE := INITIAL_VALUE;
    signal SIGNAL_NAME : SIGNAL_TYPE;
	      \end{verbatim}
	\item Esempio di un \textit{signal}
	      \begin{verbatim}
    signal s1 : std_logic := '1';
    signal s2 : std_logic;
	      \end{verbatim}
	\item \textbf{Il \textit{signal} si dichiara all'interno dell'\textit{architecture}, prima del \textit{begin}}
\end{itemize}

\subsection{Component}
\begin{itemize}
	\item Per utilizzare i moduli all'interno del design, questi devono essere prima \textit{dichiarati} e poi \textit{istanziati}
	\begin{itemize}
	\item La \textit{dichiarazione} deve avvenire all'interno dell'\textit{architecture}
	\item L'\textit{istanziamento} deve avvenire dopo il \textit{begin}
\end{itemize}
\item Struttura di una \textit{dichiarazione} :
\begin{verbatim}
    component ENTITY NAME is
    port (
        PORT_NAME : PORT_MODE PORT_TYPE;
        PORT_NAME : PORT_MODE PORT_TYPE
        ...
        );
    end component;
\end{verbatim}
	\item \textbf{Non ci vuole il \textit{punto e virgola} nella dichiarazione dell'ultima porta}
	\item La struttura delle porte del \textit{componente dichiarato} deve essere identica a quella \textit{dell'entity originale.}
\item Struttura di un \textit{istanziamento}:
\begin{verbatim}
    INSTANTIATION_NAME : INSTANTIATED_NAME
    port map (
        PORT_NAME => SIGNAL,
        ...
    );
\end{verbatim}
\item \textbf{Non ci vuole la \textit{virgola} nel map dell'ultima porta}
\item Esempio di una \textit{dichiarazione}:
\begin{verbatim}
    component or2 is
    port (
        a : in std_logic;
        b : in std_logic;
        c : out std_logic
    );
    end component;
\end{verbatim}
\item Esempio di un \textit{istanziamento}:
\begin{verbatim}
    or2_inst1 : or2
    port map (
        a => a1,
        b => a0,
        c => n1
    );
\end{verbatim}
\end{itemize}

\subsection{Generic}
\begin{itemize}
	\item I \textit{generic} servono a passare informazioni all'\textit{entity}
	\item Non possono essere modificati durante l'esecuzione
	\item Specificano parametri, ad esempio larghezza di vettori
	\item Si usano per rendere scalabile un sistema
	\item Possono essere considerate come \textit{costanti}
	\item Struttura di un \textit{generic}
	      \begin{verbatim}
    Generic(
        GENRIC_NAME_1 : GENERIC_TYPE := INITIAL_VALUE;
        ...
        GENRIC_NAME_2 : GENERIC_TYPE := INITIAL_VALUE
    );
	      \end{verbatim}
	\item Esempio di un \textit{generic}
	      \begin{verbatim}
    entity my_entity is
    Generic (
        VECTOR_WIDTH : integer := 16;
        XOR_CHECK : boolean := false
    );
    Port (
        a : in std_logic_vector(VECTOR_WIDTH-1 downto 0);
        b : out std_logic_vector(VECTOR_WIDTH-1 downto 0)
    );
    end my_entity;
	      \end{verbatim}
	\item \textbf{Vanno usati nella \textit{entity} prima della funzione \textit{port}}
\end{itemize}

\subsection{Attributes} \label{attributes}
\begin{itemize}
	\item Permettono di ricavare informazioni su oggetti del nostro sistema
	\item \textit{Attributes} più usati:
	      \begin{itemize}
	      	\item LEFT/RIGHT
	      	\item HIGH/LOW
	      	\item LENGTH
	      	\item RANGE/REVERSE\_RANGE
	      \end{itemize}
	      
\newpage

	\item Struttura di un \textit{attribute}
	      \begin{verbatim}
    object'attribute_name
	      \end{verbatim}
	\item Esempi di \textit{attribute}
	      \begin{verbatim}
    signal vect : std_logic_vector(4 to 16);
    vect'LEFT => 4
    vect'RIGHT => 4
    vect'HIGH => 16
    vect'LOW => 4
    vect'LENGTH => 13
    vect'RANGE => (4 to 16)
    vect'REVERSE_RANGE => (16 downto 4)
	      \end{verbatim}
\end{itemize}




\subsection{Aggregate} \label{aggregates}
\begin{itemize}
	\item Definisce un vettore tramite composizione
	\item A differenza della concatenazione, la lunghezza del vettore non è definita a priori

	\item Esempi di \textit{aggregate}
	      \begin{verbatim}
    signal vect_1 : std_logic_vector(7 downto 0);
    signal vect_2 : std_logic_vector(7 downto 0);
        
    vect1 <= (7 => '0', 6 => vect2(3), 3|4|2 => '1', Others => '0');
    vect1 <= (vect2(3 downto 0), vect2(7 downto 4));
    vect1 <= (Others => '0');
    vect1 <= (7 downto 4 => '0', 3 downto 0 => '1');
	      \end{verbatim}
\end{itemize}

\subsection{When/Else}
\begin{itemize}
	\item Assegna il valore ad un \textit{signal} condizionatamente ad un altro \textit{signal,} \textbf{ma solo al di fuori di un process}
	\item Tutte le possibili condizioni vanno esplicitate, usando la \textit{keyword else}
	\item Non bisogna usare le virgole dopo gli else
	\item Struttura di un \textit{when/else}
	      \begin{verbatim}
    SIGNAL_1 <= VALUE when SIGNAL_2 = VALUE else
                ...
                else VALUE;
	      \end{verbatim}
	      
\newpage	      
	      
	\item Esempio di \textit{when/else}
	      \begin{verbatim}
    a, b signals
    b <= "1000" when a = "00" else
         "0100" when a = "01" else
         "0010" when a = "10" else
         "0001" when a = "11";
             
    b <= "1000" when a = "00" else
         "0100" when a = "01" else
         "0000";
	      \end{verbatim}
\end{itemize}

\subsection{With/Select}
\begin{itemize}
	\item Assegna il valore ad un \textit{signal} condizionatamente ad un altro \textit{signal, } \textbf{ma solo al di fuori di un process}
	\item Tutte le possibili condizioni vanno esplicitate, usando la \textit{keyword else}
	\item Servono le virgole a fine riga
	\item Struttura di un \textit{with/select}
	\begin{verbatim}
    with SIGNAL_1 select SIGNAL_2 <= VALUE when VALUE;
                                         ...
	      \end{verbatim}
	\item Esempio di \textit{with/select}
	      \begin{verbatim}
    a, b signals
    with a select b <= "1000" when "00";
                       "0100" when "01";
                       "0010" when "10";
                       "0001" when "11";
      
    with a select b <= "1000" when "00";
                       "0100" when "01";
                       "0000" when Others;
	      \end{verbatim}
\end{itemize}


 \newpage

\section{Generate statement}
\begin{itemize}
	\item Servono ad istanziare un numero di componenti non definito staticamente
	\item Il numero di \textit{component} istanziati può essere in funzione dei \textit{generic} 
	\item I \textit{generate statements} possono essere annidati
	\item \textbf{Vanno usati dentro l'\textit{architecture}, dopo il  \textit{begin}}
\end{itemize}
\subsection{If ... Generate}
\begin{itemize}
	\item Tramite il costrutto \textit{If ... Generate} è possibile includere determinate dichiarazioni in funzione del valore assunto da \textit{generic} o \textit{costanti}
	\item Il costrutto \textit{If ... Generate} non ha il costrutto \textit{else}
	\item Struttura di un \textit{If ... Generate}
	\begin{verbatim}
    	LABEL : if GENERIC == VALUE generate
        	STATEMENT
    	end generate;
    \end{verbatim}
	\item Esempio di un \textit{If ... Generate}
	      \begin{verbatim}
	    architecture Behavioral of test is
        begin
            NULL_GEN : if NULL_VALUE = true generate
                c <= (Others => '0');
            end generate;
        
            NO_NULL_GEN : if NULL_VALUE = false generate
                c <= a;
            end generate;
        end Behavioral;
	      \end{verbatim}
\end{itemize}

\subsection{For ... Generate}
\begin{itemize}
	\item Tramite il costrutto \textit{For ... Generate} è possibile includere determinate dichiarazioni ricorsivamente, come in un ciclo \textit{for}
	\item Struttura di un \textit{For ... Generate}
	      \begin{verbatim}
    LABEL : for VARIABLE in RANGE generate
        STATEMENT
    end generate;
	      \end{verbatim}
	      
\newpage
	      
	\item Esempio di un \textit{For ... Generate}
	      \begin{verbatim}
    architecture Behavioral of test is
    begin
        LOOP_GEN_1 : for I in 0 to 3 generate
            c(I) <= a(0);
        end generate;
        
        LOOP_GEN_2 : for I in 4 to 7 generate
            c(I) <= a(1);
        end generate;
    end Behavioral;
	      \end{verbatim}
\end{itemize}

\newpage

\section{Process}
\begin{itemize}
	\item Il \textit{process} in VHDL è una struttura particolare usata per descrivere il comportamento di un aparte di circuito
	\item \textbf{Il comportamento è diverso da un linguaggio di programmazione classico}
	\item Durante la simulazione, un \textit{process} viene letto ed eseguito sequenzialmente dal simulatore
	\item \textbf{I \textit{process} vanno messi all'iterno del \textit{begin}}
	\item Differenza tra \textit{sintesi} e \textit{simulazione}
	      \begin{itemize}
	      	\item Sintesi: \textit{il process viene sintetizzato per ottenere un Hardware con le stesse proprietà}
	      	\item Simulazione: \textit{il process viene letto da una macchina temporale che valuta riga per riga}
	      \end{itemize}
	\item Struttura di un \textit{process}
	      \begin{verbatim}
    PROCESS_NAME process (SENSITIVITY_LIST)
        TYPE_DECLARATIONS
        CONSTANT_DECLARATIONS
        VARIABLE_DECLARATIONS
        SUBPROGRAM_DECLARATIONS
    begin
        SEQUENTIAL_STATEMENTS
    end process PROCESS_NAME;
	      \end{verbatim}
\end{itemize}
\subsection{Struttura di un process}
\subsubsection{Infinite loop}
\begin{itemize}
	\item In un \textit{process} il codice viene letto sequenzialmente.
	\item Quando il \textit{process} raggiunge l'ultima riga, riparte dall'inizio
	\item \textbf{Senza nessun tipo di interruzione si crea un \textit{infinite loop process}}
	\begin{itemize}
		\item \textbf{\`E normalmente un comportamento non voluto}
	\end{itemize}
\end{itemize}
\subsubsection{Wait statement}
\begin{itemize}
	\item Il \textit{wait statement} temporizza il circuito, fermandone l'esecuzione in determinate condizioni
	\item Struttura di un \textit{wait statement}
	      \begin{verbatim}
    LABEL wait SENSITIVITY_CLAUSE CONDITION_CLAUSE;
	      \end{verbatim}
	      
\newpage	      
	      
	\item Esempio di un \textit{wait statement}
	      \begin{verbatim}
    wait for 10 ns;              -- Specifica un tempo di timeout
    wait until clk='1';          -- Condizione booleana
    wait until A>B and S1 or S2; -- Condizione booleana
    wait on sig1, sig2;          -- Sensitivity clause, qualsiasi cambio
                                 -- in uno dei segnali ferma la pausa
    wait;                        -- Pausa per semprea
	      \end{verbatim} 
\end{itemize}
\subsubsection{Sensitivity list}
\begin{itemize}
	\item La \textit{sensitivity list}, insieme agli \textit{wait statements}, serve a controllare il flusso del programma
	\item Funge da \textit{lista di attivazione} del processo, poiché questo rimane dormiente finché un segnale non lo risveglia cambiando di valore
	\item Nella \textit{sensitivity list} vanno aggiunti tutti i segnali presenti nella parte destra delle assegnazioni dei vari costrutti condizionali
\end{itemize}
\subsubsection{Declaration region}
\begin{itemize}
	\item All'interno di un \textit{process} sono visibili tutte le dichiarazioni presenti nella \textit{entity} e \textit{architecture}, come \textit{signals} e \textit{begin}
	\item La \textit{declaration region} è nell'area compresa tra \textit{process} e \textit{begin}
	\item Dentro la \textit{declaration region} è possibile dichiarare nuove risorse (le \textit{variabili}). Non esistono \textit{signals} locali
	\item Esempio di \textit{declaration region}
	      \begin{verbatim}
	    process(clk, reset)
	        variable local_count : count'base
	    begin
        	...
	    end process;
	      \end{verbatim}
\end{itemize}
\subsubsection{Sequential Statement Region}
\begin{itemize}
	\item Nella regione di \textit{sequential statement} viene inserita tutta la descrizione del comportamento del \textit{process}
	\item La lettura del codice avviene in maniera \textit{sequenziale}
	
	\newpage
	
	\item Esempio di \textit{sequential statement region}
		\begin{verbatim}
    process (...)
        ...
    begin
        local_count := count;
        if reset = '1' then
            local_count <= 0;
        elsif rising_edge(clk) then
            local_count <= local_count + 1;
        end if;
        count <= local_count;
    end process;
	      \end{verbatim}
\end{itemize}
\subsection{Sequential statements}
\begin{itemize}
	\item Questi \textit{statements} sono usati nei \textit{process}
	\item Lista di \textit{statements}
	      \begin{itemize}
	      	\item wait statement
	      	\item assertion statement
	      	\item report statement
	      	\item signal assignment statement
	      	\item procedure call statement
	      	\item if statement
	      	\item case statement
	      	\item loop statement
	      	\item next statement
	      	\item exit statement
	      	\item return statement
	      	\item null statement
	      \end{itemize}
\end{itemize}
\subsubsection{If statement}
\begin{itemize}
	\item Struttura dell'\textit{if statement}
	      \begin{verbatim}
    LABEL if CONDITION1 then
        ...
    elsif CONDITION2 then
        ...
    else
        ...
    end if LABEL;
	      \end{verbatim}
	\item Esempio di \textit{if statement}
	      \begin{verbatim}
    if a=b then
        c := a;
    elsif b<c then
        d := b;
        b := c;
    else
        c := 0;
    end if;
	      \end{verbatim}
\end{itemize}
\subsubsection{Case statement}
\begin{itemize}
	\item \'E necessario specificare \textbf{sempre} il caso \textit{Others}
	\item Struttura dell'\textit{Case statement}
	      \begin{verbatim}
    LABEL case EXPRESSION is
        when choice1 =>
            ...
        when choice2 =>
            ...
        when others =>
            ...
    end case LABEL
	      \end{verbatim}
	\item Esempio di \textit{case statement}
	      \begin{verbatim}
    case my_val is
        when 1 =>
            a:=b;
        when 3 =>
            c:=d;
        when Others =>
            b:=c;
    end case
	      \end{verbatim}
\end{itemize}

\newpage 

\subsubsection{Loop statement}
\begin{itemize}
	\item Struttura dell'\textit{Loop statement}
	      \begin{verbatim}
    LABEL loop
        ... --use exit statement to get out
    end loop LABEL;
    
    LABEL for VARIABLE in RANGE loop
        ...
    end loop LABEL;
    
    LABEL while CONDITION loop
        ...
    end loop LABEL;
	      \end{verbatim}
	\item Esempio di \textit{loop statement}
	      \begin{verbatim}
    loop
        input_something;
        exit when end_file;
    end loop;
    
    for I in 1 to 10 loop
        AA(I) := 0;
    end loop;
    
    while not end_file loop
        input_something;
    end loop;
	      \end{verbatim}
\end{itemize}

\subsection{Commit}
\begin{itemize}
	\item Il \textit{commit} indica quando il valore delli \textit{signals} e delle \textit{variables} vengono aggiornati nel sistema
\end{itemize}
\subsubsection{Signal commit}
Nei \textit{process} i segnali hanno un comportamento particolare
\begin{itemize}
	\item Il valore dei segnali rimane invariato tra due \textit{wait statement}
	\item Il valore scritto nei segnali viene aggiornato solo quanto il flusso raggiunge uno \textit{wait statement}
\end{itemize}
\subsubsection{Variable commit}
\begin{itemize}
	\item Le \textit{variables} hanno visibilità ristretta nel \textit{process} che le ha dichiarate
	\item Le \textit{variables} vengono aggiornate istantaneamente
	\item L'assegnazione delle \textit{variables} avviene con \(=:\)
\end{itemize}

\newpage

\section{Custom types}
\begin{itemize}
	\item Nel VHDL sono disponibili molti tipi di dato base. Per creare nuovi tipi di dato personalizzati si usano le seguenti keywords
	      \begin{itemize}
	      	\item Type \textit{tipi}
	      	\item Subtype \textit{sottotipi}
	      \end{itemize}
	\item \textbf{Si usano nell'architecture prima del \textit{begin}}
\end{itemize}
\subsection{Sottotipi}
\begin{itemize}
	\item Un \textit{subtype} è un sottoinsieme di un tipo già esistente \textit{(per esempio, integer o std\_logic\_vector)}
	\item I nuovi tipi creati possono essere assegnati anche al tipo originale
	\item Struttura di un \textit{subtype}
	      \begin{verbatim}
	    subtype SUBTYPE_NAME TYPE STATEMENTS;
	      \end{verbatim}
	\item Esempi di \textit{subtypes}
	      \begin{verbatim}
	    subtype short integer range 0 to 255;
		
	    subtype nib is std_logic_vector(3 downto 0);
	    subtype byte is std_logic_vector(7 downto 0);
		
	    signal myByte : byte;
	    signal myVect : std_logic_vector(7 downto 0);
		
	    myVect <= mybyte;
	      \end{verbatim}
\end{itemize}
\subsection{Tipi}
\subsubsection{Tipi enumerati}
\begin{itemize}
	\item \textit{Tipi} di dato che rappresentano un set finito di stati diversi
	\item Struttura di un \textit{tipo enumerato}
	      \begin{verbatim}
	    type TYPE_NAME is (ELEMENT, ELEMENT, ...);
	      \end{verbatim}
	\item Esempi di \textit{tipi enumerati}
	      \begin{verbatim}
	    type MyBit_type is (L, H);
	    type MyState_type is (init, waiting, working, done);
	      \end{verbatim}
\end{itemize}

\subsubsection{Tipi record}
\begin{itemize}
	\item \textit{Tipi} di dato composto da diversi sotto oggetti
	\item Struttura di un \textit{tipo record}
	      \begin{verbatim}
	    type TYPE_NAME is record
		    ELEMENT_NAME : element type;
		    ...
	    end record TYPE_NAME;
	      \end{verbatim}
	\item Esempi di \textit{tipi record}
	      \begin{verbatim}
	    type Operation is record
		    OpCode : Bit_Vector(3 downto 0);
		    Op1, Op2, Res : RegName;
	    end record;
	      \end{verbatim}
\end{itemize}

\subsubsection{Tipi array}
\begin{itemize}
	\item \textit{Tipi} di dato composto da multipli elementi dello stesso tipo
	\item Struttura di un \textit{tipo array}
	      \begin{verbatim}
	    type TYPE_NAME is array (RANGE) of ELEMENT_TYPE;
	      \end{verbatim}
	\item Esempi di \textit{tipi record}
	      \begin{verbatim}
	    type nibble is array (3 downto 0) of std_ulogic;
	    type RAM is array (0 to 31) of integer range 0 to 255;
	      \end{verbatim}
\end{itemize}

\paragraph{Array 1D}
\begin{itemize}
	\item \textit{Array} in cui è definito un singolo range
	\item \`E possibile creare \textit{nested 1D array} in cui ogni eleento di ogni cella è a sua volta un array
	\item Struttura di un \textit{array 1D}
	\begin{verbatim}
	    type TYPE_NAME is array (RANGE) of ANOTHER_TYPE;
	\end{verbatim}
	\item Esempi di \textit{array 1D}
	\begin{verbatim}
	    type MyNestedArray is array (0 to 10) of std_logic_vector(7 downto 0);
	    signal ciao : MyNestedArray := (Others => (Others => '0'));
	    ciao(0)(0) <= '1';
	    ciao(0) <= "01010101";
	\end{verbatim}
\end{itemize}

\paragraph{Array multidimensionali}
\begin{itemize}
	\item \textit{Array} con più indici
	\item Utili quando la grandezza da rappresentare è a sua volta a più dimensioni \textit{(es. immagini RGB)}
	\item Struttura di un \textit{array multidimensionale}
	\begin{verbatim}
	    type ARRAY_TYPE is array (RANGE, RANGE, ...) of ANOTHER TYPE
	\end{verbatim}
	\item Esempio di \textit{array multidimensionale}
	\begin{verbatim}
	    type RGB_Type is record
	        r_ch : unsigned(7 downto 0);
	        g_ch : unsigned(7 downto 0);
	        b_ch : unsigned(7 downto 0);
	    end record;

	    type img_type is array (0 to 15, 0 to 15) of RGB_Type;
		
	    signal immagine : img_type := (Others => (Others => ( 
		                                    r_ch => to_unsigned(100, 8),
		                                    g_ch => to_unsigned(100, 8),
		                                    b_ch => to_unsigned(100, 8)
		                               )));
		
	    immagine(0, 0) <= ( 
	                        r_ch => to_unsigned(100, 8),
	                        g_ch => to_unsigned(100, 8),
	                        b_ch => to_unsigned(100, 8)
		                   );
	\end{verbatim}
\end{itemize}

\paragraph{Constrained array}
\begin{itemize}
	\item \textit{Tipo di array} in cui è definita la dimensione in fase di definizione
	\item Ogni segnale dichiarato con questo tipo sarà della stessa dimensione
	\item Può essere anche un \textit{subtype}
	\item Struttura di un \textit{constrained array}
	\begin{verbatim}
	    type TYPE_NAME is array (RANGE) of ANOTHER_TYPE;
	\end{verbatim}
	\item Esempi di \textit{constrained array}
	\begin{verbatim}
	    type MySimpleArray is array (0 to 10) of Another_type;
	    
	    type img_type is array (0 to 15, 0 to 15) of RGB_Type;
	    subtype byte_type is std_logic_vector(7 downto 0);
	\end{verbatim}
\end{itemize}

\paragraph{Unconstrained array}
\begin{itemize}
	\item \textit{Tipo di array} in cui \textbf{non è} definita la dimensione in fase di definizione
	\item Ogni segnale dichiarato con questo tipo potrà avere una dimensione diversa
	\item Può essere anche un \textit{subtype}
	\item Nel momento in cui si dichiara il segnale si deve specificare un \textit{range} per definirne la grandezza. Il \textit{range} può non essere un \textit{integer} ma anche un \textit{natural}
	\begin{verbatim}
	    type TYPE_NAME is array (TYPE range<>, TYPE range<>, ...) of ANOTHER_TYPE;
	\end{verbatim}
	\item Esempi di \textit{unconstrained array}
	\begin{verbatim}
	    type my1DArray_type is array (integer range <>, integer range <>) 
	        of std_logic_vector(7 downto 0);
	        
	    signal my1DArray : my1DArray_type(1 to 3);
		
		
	    type my2DArray_type is array (integer range<>, integer range<>) 
    	    of std_logic_vector(7 downto 0);
    	    
	    signal my2DArray : my2DArray_type(1 to 3, 5 downto 0);
	\end{verbatim}
\end{itemize}
\end{document}
